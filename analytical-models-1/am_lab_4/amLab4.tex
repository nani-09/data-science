\PassOptionsToPackage{unicode=true}{hyperref} % options for packages loaded elsewhere
\PassOptionsToPackage{hyphens}{url}
%
\documentclass[]{article}
\usepackage{lmodern}
\usepackage{amssymb,amsmath}
\usepackage{ifxetex,ifluatex}
\usepackage{fixltx2e} % provides \textsubscript
\ifnum 0\ifxetex 1\fi\ifluatex 1\fi=0 % if pdftex
  \usepackage[T1]{fontenc}
  \usepackage[utf8]{inputenc}
  \usepackage{textcomp} % provides euro and other symbols
\else % if luatex or xelatex
  \usepackage{unicode-math}
  \defaultfontfeatures{Ligatures=TeX,Scale=MatchLowercase}
\fi
% use upquote if available, for straight quotes in verbatim environments
\IfFileExists{upquote.sty}{\usepackage{upquote}}{}
% use microtype if available
\IfFileExists{microtype.sty}{%
\usepackage[]{microtype}
\UseMicrotypeSet[protrusion]{basicmath} % disable protrusion for tt fonts
}{}
\IfFileExists{parskip.sty}{%
\usepackage{parskip}
}{% else
\setlength{\parindent}{0pt}
\setlength{\parskip}{6pt plus 2pt minus 1pt}
}
\usepackage{hyperref}
\hypersetup{
            pdftitle={am\_lab\_4},
            pdfauthor={Pradeep Paladugula},
            pdfborder={0 0 0},
            breaklinks=true}
\urlstyle{same}  % don't use monospace font for urls
\usepackage[margin=1in]{geometry}
\usepackage{color}
\usepackage{fancyvrb}
\newcommand{\VerbBar}{|}
\newcommand{\VERB}{\Verb[commandchars=\\\{\}]}
\DefineVerbatimEnvironment{Highlighting}{Verbatim}{commandchars=\\\{\}}
% Add ',fontsize=\small' for more characters per line
\usepackage{framed}
\definecolor{shadecolor}{RGB}{248,248,248}
\newenvironment{Shaded}{\begin{snugshade}}{\end{snugshade}}
\newcommand{\AlertTok}[1]{\textcolor[rgb]{0.94,0.16,0.16}{#1}}
\newcommand{\AnnotationTok}[1]{\textcolor[rgb]{0.56,0.35,0.01}{\textbf{\textit{#1}}}}
\newcommand{\AttributeTok}[1]{\textcolor[rgb]{0.77,0.63,0.00}{#1}}
\newcommand{\BaseNTok}[1]{\textcolor[rgb]{0.00,0.00,0.81}{#1}}
\newcommand{\BuiltInTok}[1]{#1}
\newcommand{\CharTok}[1]{\textcolor[rgb]{0.31,0.60,0.02}{#1}}
\newcommand{\CommentTok}[1]{\textcolor[rgb]{0.56,0.35,0.01}{\textit{#1}}}
\newcommand{\CommentVarTok}[1]{\textcolor[rgb]{0.56,0.35,0.01}{\textbf{\textit{#1}}}}
\newcommand{\ConstantTok}[1]{\textcolor[rgb]{0.00,0.00,0.00}{#1}}
\newcommand{\ControlFlowTok}[1]{\textcolor[rgb]{0.13,0.29,0.53}{\textbf{#1}}}
\newcommand{\DataTypeTok}[1]{\textcolor[rgb]{0.13,0.29,0.53}{#1}}
\newcommand{\DecValTok}[1]{\textcolor[rgb]{0.00,0.00,0.81}{#1}}
\newcommand{\DocumentationTok}[1]{\textcolor[rgb]{0.56,0.35,0.01}{\textbf{\textit{#1}}}}
\newcommand{\ErrorTok}[1]{\textcolor[rgb]{0.64,0.00,0.00}{\textbf{#1}}}
\newcommand{\ExtensionTok}[1]{#1}
\newcommand{\FloatTok}[1]{\textcolor[rgb]{0.00,0.00,0.81}{#1}}
\newcommand{\FunctionTok}[1]{\textcolor[rgb]{0.00,0.00,0.00}{#1}}
\newcommand{\ImportTok}[1]{#1}
\newcommand{\InformationTok}[1]{\textcolor[rgb]{0.56,0.35,0.01}{\textbf{\textit{#1}}}}
\newcommand{\KeywordTok}[1]{\textcolor[rgb]{0.13,0.29,0.53}{\textbf{#1}}}
\newcommand{\NormalTok}[1]{#1}
\newcommand{\OperatorTok}[1]{\textcolor[rgb]{0.81,0.36,0.00}{\textbf{#1}}}
\newcommand{\OtherTok}[1]{\textcolor[rgb]{0.56,0.35,0.01}{#1}}
\newcommand{\PreprocessorTok}[1]{\textcolor[rgb]{0.56,0.35,0.01}{\textit{#1}}}
\newcommand{\RegionMarkerTok}[1]{#1}
\newcommand{\SpecialCharTok}[1]{\textcolor[rgb]{0.00,0.00,0.00}{#1}}
\newcommand{\SpecialStringTok}[1]{\textcolor[rgb]{0.31,0.60,0.02}{#1}}
\newcommand{\StringTok}[1]{\textcolor[rgb]{0.31,0.60,0.02}{#1}}
\newcommand{\VariableTok}[1]{\textcolor[rgb]{0.00,0.00,0.00}{#1}}
\newcommand{\VerbatimStringTok}[1]{\textcolor[rgb]{0.31,0.60,0.02}{#1}}
\newcommand{\WarningTok}[1]{\textcolor[rgb]{0.56,0.35,0.01}{\textbf{\textit{#1}}}}
\usepackage{graphicx,grffile}
\makeatletter
\def\maxwidth{\ifdim\Gin@nat@width>\linewidth\linewidth\else\Gin@nat@width\fi}
\def\maxheight{\ifdim\Gin@nat@height>\textheight\textheight\else\Gin@nat@height\fi}
\makeatother
% Scale images if necessary, so that they will not overflow the page
% margins by default, and it is still possible to overwrite the defaults
% using explicit options in \includegraphics[width, height, ...]{}
\setkeys{Gin}{width=\maxwidth,height=\maxheight,keepaspectratio}
\setlength{\emergencystretch}{3em}  % prevent overfull lines
\providecommand{\tightlist}{%
  \setlength{\itemsep}{0pt}\setlength{\parskip}{0pt}}
\setcounter{secnumdepth}{0}
% Redefines (sub)paragraphs to behave more like sections
\ifx\paragraph\undefined\else
\let\oldparagraph\paragraph
\renewcommand{\paragraph}[1]{\oldparagraph{#1}\mbox{}}
\fi
\ifx\subparagraph\undefined\else
\let\oldsubparagraph\subparagraph
\renewcommand{\subparagraph}[1]{\oldsubparagraph{#1}\mbox{}}
\fi

% set default figure placement to htbp
\makeatletter
\def\fps@figure{htbp}
\makeatother


\title{am\_lab\_4}
\author{Pradeep Paladugula}
\date{2/11/2020}

\begin{document}
\maketitle

\hypertarget{r-markdown}{%
\subsection{R Markdown}\label{r-markdown}}

\begin{Shaded}
\begin{Highlighting}[]
\KeywordTok{download.file}\NormalTok{(}\StringTok{"http://www.openintro.org/stat/data/bdims.RData"}\NormalTok{, }\DataTypeTok{destfile =} \StringTok{"bdims.RData"}\NormalTok{)}
\KeywordTok{load}\NormalTok{(}\StringTok{"bdims.RData"}\NormalTok{)}
\KeywordTok{head}\NormalTok{(bdims)}
\end{Highlighting}
\end{Shaded}

\begin{verbatim}
##   bia.di bii.di bit.di che.de che.di elb.di wri.di kne.di ank.di sho.gi che.gi
## 1   42.9   26.0   31.5   17.7   28.0   13.1   10.4   18.8   14.1  106.2   89.5
## 2   43.7   28.5   33.5   16.9   30.8   14.0   11.8   20.6   15.1  110.5   97.0
## 3   40.1   28.2   33.3   20.9   31.7   13.9   10.9   19.7   14.1  115.1   97.5
## 4   44.3   29.9   34.0   18.4   28.2   13.9   11.2   20.9   15.0  104.5   97.0
## 5   42.5   29.9   34.0   21.5   29.4   15.2   11.6   20.7   14.9  107.5   97.5
## 6   43.3   27.0   31.5   19.6   31.3   14.0   11.5   18.8   13.9  119.8   99.9
##   wai.gi nav.gi hip.gi thi.gi bic.gi for.gi kne.gi cal.gi ank.gi wri.gi age
## 1   71.5   74.5   93.5   51.5   32.5   26.0   34.5   36.5   23.5   16.5  21
## 2   79.0   86.5   94.8   51.5   34.4   28.0   36.5   37.5   24.5   17.0  23
## 3   83.2   82.9   95.0   57.3   33.4   28.8   37.0   37.3   21.9   16.9  28
## 4   77.8   78.8   94.0   53.0   31.0   26.2   37.0   34.8   23.0   16.6  23
## 5   80.0   82.5   98.5   55.4   32.0   28.4   37.7   38.6   24.4   18.0  22
## 6   82.5   80.1   95.3   57.5   33.0   28.0   36.6   36.1   23.5   16.9  21
##    wgt   hgt sex
## 1 65.6 174.0   1
## 2 71.8 175.3   1
## 3 80.7 193.5   1
## 4 72.6 186.5   1
## 5 78.8 187.2   1
## 6 74.8 181.5   1
\end{verbatim}

\begin{Shaded}
\begin{Highlighting}[]
\CommentTok{#Since males and females tend to have different body dimensions, it will be useful to create two additional data sets: }
\CommentTok{#one with only men and another with only women.``}
\NormalTok{mdims <-}\StringTok{ }\KeywordTok{subset}\NormalTok{(bdims, sex }\OperatorTok{==}\StringTok{ }\DecValTok{1}\NormalTok{)}
\NormalTok{fdims <-}\StringTok{ }\KeywordTok{subset}\NormalTok{(bdims, sex }\OperatorTok{==}\StringTok{ }\DecValTok{0}\NormalTok{)}
\end{Highlighting}
\end{Shaded}

.Now let's consider some of the other variables in the body dimensions
data set. Using the figures at the end of the exercises, match the
histogram to its normal probability plot. All of the variables have been
standardized (first subtract the mean, then divide by the standard
deviation), so the units won't be of any help. If you are uncertain
based on these figures, generate the plots in R to check. (B,C,D)

\begin{enumerate}
\def\labelenumi{\alph{enumi}.}
\item
  The histogram for female biiliac (pelvic) diameter ( bii.di ) belongs
  to normal probability plot letter \textbf{B}.
\item
  The histogram for female elbow diameter ( elb.di ) belongs to normal
  probability plot letter \textbf{C}.
\item
  The histogram for general age ( age ) belongs to normal probability
  plot letter \textbf{D}.
\item
  The histogram for female chest depth ( che.de ) belongs to normal
  probability plot letter \textbf{A}.
\end{enumerate}

.Note that normal probability plots C and D have a slight stepwise
pattern. Why do you think this is the case?

\begin{verbatim}
Solution: Probably data might be predictive data (choose randomly) which are picked randomly may cause the graph to get the stepwise pattern.
\end{verbatim}

.As you can see, normal probability plots can be used both to assess
normality and visualize skewness. Make a normal probability plot for
female knee diameter ( kne.di ). Based on this normal probability plot,
is this variable left skewed, symmetric, or right skewed? Use a
histogram to confirm your findings.

\begin{verbatim}
Solution: Form the denity distribution plot and normal distribution plot, it is clear that the varibale is right skewed. 
\end{verbatim}

\begin{Shaded}
\begin{Highlighting}[]
\NormalTok{fkneedim <-}\StringTok{ }\NormalTok{fdims}\OperatorTok{$}\NormalTok{kne.di}
\NormalTok{meanfknee <-}\StringTok{ }\KeywordTok{mean}\NormalTok{(fkneedim)}
\NormalTok{sdfknee <-}\StringTok{ }\KeywordTok{sd}\NormalTok{(fkneedim)}
\KeywordTok{hist}\NormalTok{(fkneedim, }\DataTypeTok{probability =} \OtherTok{TRUE}\NormalTok{)}
\NormalTok{x <-}\StringTok{ }\DecValTok{14}\OperatorTok{:}\DecValTok{26}
\NormalTok{y <-}\StringTok{ }\KeywordTok{dnorm}\NormalTok{(}\DataTypeTok{x =}\NormalTok{ x, }\DataTypeTok{mean =}\NormalTok{ meanfknee, }\DataTypeTok{sd =}\NormalTok{ sdfknee)}
\KeywordTok{lines}\NormalTok{(}\DataTypeTok{x =}\NormalTok{ x, }\DataTypeTok{y =}\NormalTok{ y, }\DataTypeTok{col =} \StringTok{'blue'}\NormalTok{)}
\end{Highlighting}
\end{Shaded}

\includegraphics{amLab4_files/figure-latex/Probability for feamale knee diameter-1.pdf}

\begin{Shaded}
\begin{Highlighting}[]
\KeywordTok{qqnorm}\NormalTok{(fdims}\OperatorTok{$}\NormalTok{kne.di)}
\KeywordTok{qqline}\NormalTok{(fdims}\OperatorTok{$}\NormalTok{kne.di)}
\end{Highlighting}
\end{Shaded}

\includegraphics{amLab4_files/figure-latex/femal knee diameter-1.pdf}

Normal Probablity Plots:

\begin{Shaded}
\begin{Highlighting}[]
\KeywordTok{qqnorm}\NormalTok{(fdims}\OperatorTok{$}\NormalTok{bii.di)}
\KeywordTok{qqline}\NormalTok{(fdims}\OperatorTok{$}\NormalTok{bii.di)}
\end{Highlighting}
\end{Shaded}

\includegraphics{amLab4_files/figure-latex/B-1.pdf}

\begin{Shaded}
\begin{Highlighting}[]
\KeywordTok{qqnorm}\NormalTok{(fdims}\OperatorTok{$}\NormalTok{elb.di)}
\KeywordTok{qqline}\NormalTok{(fdims}\OperatorTok{$}\NormalTok{elb.di)}
\end{Highlighting}
\end{Shaded}

\includegraphics{amLab4_files/figure-latex/C-1.pdf}

\begin{Shaded}
\begin{Highlighting}[]
\KeywordTok{qqnorm}\NormalTok{(fdims}\OperatorTok{$}\NormalTok{age)}
\KeywordTok{qqline}\NormalTok{(fdims}\OperatorTok{$}\NormalTok{age)}
\end{Highlighting}
\end{Shaded}

\includegraphics{amLab4_files/figure-latex/D-1.pdf}

\begin{Shaded}
\begin{Highlighting}[]
\KeywordTok{qqnorm}\NormalTok{(fdims}\OperatorTok{$}\NormalTok{che.de)}
\KeywordTok{qqline}\NormalTok{(fdims}\OperatorTok{$}\NormalTok{che.de)}
\end{Highlighting}
\end{Shaded}

\includegraphics{amLab4_files/figure-latex/A-1.pdf}

\end{document}
